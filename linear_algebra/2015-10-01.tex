\documentclass[11pt]{proc}
\usepackage[margin=0.5in]{geometry}
\usepackage[utf8]{inputenc}
\usepackage[russian]{babel}
\usepackage{amsmath}

\title{Линейная алгебра. 1 октября 2015.}

\begin{document}
\pagestyle{empty}
\maketitle
\section{Линейные пространства}
\textbf{Определение}
\begin{quote}
	Множество $L$ назыавется линейным пространством, а его элементы - векторами линейного пространства, если на этом множестве заданы:
	\begin{enumerate}
		\item{Закон сложения \[\bar{X},\bar{Y}\in L \rightarrow \bar{X}+\bar{Y}\in L\]}
		\item{Закон умножения вектора на число \[\bar{X}\in L, \lambda \rightarrow \lambda\bar{X}\in L\]}
		\item{$\forall\bar{X},\bar{Y},\bar{Z} \in L, \forall\alpha,\beta$ выполняются аксиомы:
		\begin{itemize}
			\item[а)]{Свойство коммутативности: $\bar{X}+\bar{Y}=\bar{Y}+\bar{X}$}
			\item[б)]{Свойство ассоциативности: $(\bar{X}+\bar{Y})+\bar{Z}=\bar{X}+(\bar{Y}+\bar{Z})$}
			\item[в)]{Свойство дистрибутивности: $\alpha(\bar{X}+\bar{Y})=\alpha\bar{X}+\alpha\bar{Y}$}
			\item[г)]{$(\alpha+\beta)\bar{X}=\alpha\bar{X}+\beta\bar{Y}$}
			\item[д)]{$\alpha(\beta\bar{X})=(\alpha\beta)\bar{X}$}
			\item[е)]{$\forall x \in L \quad \bar{x}+0=\bar{x}$}
			\item[ё)]{$\bar{x}+(-\bar{x})=0$}
			\item[ж)]{$\bar{x}\cdot1=\bar{x}$}
		\end{itemize}
		}
	\end{enumerate}
\end{quote}
\subsection{Базис и размерность линейного пространства}
\textbf{Определение}
\begin{quote}
	Базисом в линейном пространстве $L$ называют \textbf{упорядоченную} \textbf{конечную} систему векторов, если она линейно независима и любой другой вектор из $L$ есть линейная комбинация векторов базиса, а коэффициент этой линейной комбинации есть компоненты вектора по базису.
\end{quote}
	\[\bar\varepsilon = (\bar{e_1},\hdots,\bar{e_n})\textup{ - базис}\]
	\begin{multline*}\bar{X}=x_1\bar{e_1}+\hdots+x_n\bar{e_n}=\sum^n_{i=1}x_i\bar{e_i}=\\=\begin{pmatrix}\bar{e_1} & \hdots & \bar{e_n}\end{pmatrix}\begin{pmatrix}x_1 \\ \vdots \\ x_n\end{pmatrix}=\bar\varepsilon\bar{X}\textup{ - разложение по базису}\end{multline*}
	\textbf{Следствие 1}
	\begin{quote}
		Координаты столбцов суммы векторов равен сумме их координатных столбцов, и координаты столбцов произведения вектора на число равен произведению координатных столбцов на это число.
	\end{quote}
	\textbf{Следствие 2}
	\begin{quote}
		Вектора линейно зависимы тогда и только тогда, когда зависимы координатные столбцы.
	\end{quote}
	\underline{\textbf{Теорема "О базисе"}}
	\begin{quote}
		Если в ЛП $L$ существует $\bar\varepsilon$ из $n$ векторов, то любой другой вектор $\bar\varepsilon$ в этом ЛП состоит из того же числа векторов.
	\end{quote}
	\textbf{Определение}
	\begin{quote}
		ЛП, в котором $\displaystyle\exists\bar\varepsilon$ из $n$ векторов, называется $n$-мерным, а само число $n$ - размерностью ЛП - $L^n$.
	\end{quote}
	\textbf{Следствие 1}
	\begin{quote}
		В $n$-мерном ЛП каждая конечная упорядоченная система из $n$ линейно независимых векторов называется \emph{базисом}.
	\end{quote}
	\textbf{Следствие 2}
	\begin{quote}
		В $n$-мерном ЛП каждую упорядоченную линейно-независимую систему из $k<n$ векторов можно дополнить до базиса.
	\end{quote}
	\pagebreak
	\textbf{Определение}
	\begin{quote}
		Непустое множество векторов в ЛП $L^n$ называется подпространством, $L^k(k<n)$, если выполняются условия:
		\begin{enumerate}
			\item{сумма любых векторов из $L^k$ также принадлежит $L^n$}
			\item{произведения любых векторов из $L^k$ также принадлежит $L^n$}
		\end{enumerate}
	\end{quote}
	\underline{\textbf{Теорема}}
	\begin{quote}
		Если в ЛП задан базис, то координаты вектора в нем определены однозначно.
	\end{quote}
	\emph{Доказательство}
	\begin{quote}
		пусть $\displaystyle \bar{X}=\sum^n_{i=1}x_i\bar{e_i},\quad\bar{X}=\sum^n_{i=1}x'_i\bar{e_i},\\ \bar{X}=\begin{pmatrix}x_1 \\ \vdots \\ x_n\end{pmatrix}, \quad\bar{X}=\begin{pmatrix}x'_1 \\ \vdots \\ x'_n\end{pmatrix}$, тогда:
		\begin{multline*}
			\bar{X}-\bar{X}=\sum^n_{i=1}(x_i-x'_i)\bar{e_i}=0 \Rightarrow \\ \Rightarrow x_i-x'_i=0\ (i=\overline{1;n}),
			x_i=x'_i\ (i=\overline{1;n})
		\end{multline*}
		\begin{flushright}ч.т.д.\end{flushright}
	\end{quote}
\subsection{Линейное отображение и линейные преобразования ЛП}
	\textbf{Определение}
	\begin{quote}
		Отображением $A$ ЛП $L$ в $L'$ называется закон, по которому $\displaystyle\forall\bar{x}\in L$ сопоставляется $\bar{x}\in L'$,
		\[\boxed{A:L\rightarrow L'}\]
		при этом получившийся образ называется $A\bar{x}$.
	\end{quote}
	\textbf{Определение}
	\begin{quote}
		Отображение пространства на себя называется преобразованием.
	\end{quote}
	\textbf{Определение}
	\begin{quote}
		$\displaystyle A:L\rightarrow L'$ называется линейным, если $\displaystyle\forall\bar{X},\bar{Y}\in L$ и $\displaystyle\forall\lambda$ выполняются:
		\begin{enumerate}
			\item{$A(\bar{X}+\bar{Y})=A\bar{X}+A\bar{Y}$}
			\item{$A(\lambda\bar{X})=\lambda A \bar{X}$}
		\end{enumerate}
	\end{quote}
	\pagebreak
	\begin{equation}A\bar{X}=\bar{X'}:
		\begin{cases}
			\begin{matrix}
				x'_1=a_{11}x_1+\ldots+a_{1n}x_n\\
				\hdotsfor{1}\\
				x'_n=a_{n1}x_1+\ldots+a_{nn}x_n
			\end{matrix}
		\end{cases}
	\end{equation}
	$A=\begin{pmatrix}a_{11}& & \\ & \ddots & \\ & & a_{nn}\end{pmatrix}$ - матрица отображения
\subsection{Действия над преобразованием}
	\begin{enumerate}
		\item{Суммой линейных преобразований $\bar{Y}=A\bar{X}$ и $\bar{Y}=B\bar{X}$ называется $\bar{Y}=(A+B)\bar{X}$}
		\item{Произведение $\bar{Y}=A\bar{X}$ и $\bar{Y}=B\bar{X}$ называется последовательное выполнение этих преобразований $\bar{Y}=(BA)\bar{X}$}
	\end{enumerate}
	\textbf{Определение}
	\begin{quote}
		Обратное отображение обозначается, как $A^{-1}$
		\[\boxed{A^{-1}(A\bar{X})=\bar{X}}\]
	\end{quote}
	\textbf{Определение}
	\begin{quote}
		Отображение c $I$ называется тождественным, при этом:
		\[\begin{matrix}x'_1=x_1 \\ \hdotsfor{1} \\ x'_n=x_n\end{matrix}\]
	\end{quote}
	\textbf{Определение}
	\begin{quote}
		Взаимно-однозныачное линейное отображение называется \emph{афинным}, если оно задано системой (1), и матрица этого отображения является невырожденной.
	\end{quote}
	%\pagebreak
\section{Системы линейных уравнений (СЛУ)}
	\begin{equation}
		\begin{cases}
			\begin{matrix}
				a_{11}x_1+\ldots+a_{1n}x_n=b_1\\
				\hdotsfor{1}\\
				a_{n1}x_1+\ldots+a_{nn}x_n=b_n
			\end{matrix}
		\end{cases}
	\end{equation}
	
\end{document}