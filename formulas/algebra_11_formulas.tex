\documentclass{proc}
\usepackage[margin=0.5in]{geometry}
\usepackage[utf8]{inputenc}
\usepackage{amssymb}
\usepackage[russian]{babel}
%\usepackage{amsmath}

\title{Алгебра формулы}

\begin{document}
\pagestyle{empty}
\maketitle

\section{Тригонометрия}

\subsection{Основные тригонометрические тождества}
\begin{itemize}
    \item $\displaystyle\sin^2x+\cos^2x=1$
    \item $\displaystyle\tg x=\frac{\sin x}{\cos x}$
    \item $\displaystyle\ctg x=\frac{\cos x}{\sin x}$
    \item $\displaystyle\tg x\cdot\ctg x=1$
    \item $\displaystyle\tg^2x+1=\frac1{cos^2x}$
    \item $\displaystyle\ctg^2x + 1 = \frac{1}{sin^2x}$
\end{itemize}

\subsection{Обратные тригонометрические функции}
\begin{itemize}
    \item $\displaystyle\sin x = a \Leftrightarrow x=(-1)^n\arcsin(a)+\pi n, n\in\mathbb{Z}$
    \item $\displaystyle\cos x = a \Leftrightarrow x=\pm\arccos(a)+2\pi n, n\in\mathbb{Z}$
    \item $\displaystyle\tg x = a \Leftrightarrow x=\arctg(a)+\pi n, n\in\mathbb{Z}$
    \item $\displaystyle\ctg x = a \Leftrightarrow x=\arcctg(a)+\pi n, n\in\mathbb{Z}$
\end{itemize}

\subsection{Уравнения с обратными функциями}
\begin{itemize}
    \item $\displaystyle y=\arcsin x \Leftrightarrow y \in \left[-\frac \pi2;\frac \pi2\right]$
    \item $\displaystyle y=\arccos x \Leftrightarrow y \in \left[0;\pi\right]$
    \item $\displaystyle y=\arctg x \Leftrightarrow y \in \left(-\frac \pi2;\frac \pi2\right)$
    \item $\displaystyle y=\arcctg x \Leftrightarrow y \in \left(0;\pi\right)$
\end{itemize}

\subsection{Формулы двойного аргумента}
\begin{itemize}
    \item $\displaystyle\sin2x = 2\sin x\cdot\cos x$
    \item $\displaystyle\sin2x = \frac{2\tg x}{1 + \tg^2x} = \frac{2\ctg x}{1 + \ctg2x} = \frac{2}{\tg x + \ctg x}$
    \item $\displaystyle\cos2x = \cos^2x - \sin^2x = 2\cos^2x - 1 = 1 - 2\sin^2x$
    \item $\displaystyle\cos2x = \frac{1-\tg^2x}{1+\tg^2x} = \frac{\tg^2x-1}{\ctg^2x+1} = \frac{\ctg x-\tg x}{\ctg x+\tg x}$
    \item $\displaystyle\tg2x = \frac{2\tg x}{1-\tg^2x} = \frac{2\ctg x}{\ctg^2x-1} = \frac{2}{\ctg x-\tg x}$
    \item $\displaystyle\ctg2x = \frac{\ctg^2-1}{2\ctg x} = \frac{\ctg x-\tg x}{2}$
\end{itemize}

\subsection{Формулы тройного аргумента}
\begin{itemize}
    \item $\displaystyle\sin3x = 3\sin x - 4\sin^3x$
    \item $\displaystyle\cos3x = 4\cos^3x - 3\cos x$
    \item $\displaystyle\tg3x = \frac{3\tg x-\tg^3x}{1-3\tg^2x}$
    \item $\displaystyle\ctg3x = \frac{\ctg^3x-3\ctg x}{3\ctg^2x-1}$
\end{itemize}

\subsection{Формулы половинного аргумента}
\begin{itemize}
    \item $\displaystyle\sin^2\frac{x}{2} = \frac{1-\cos x}{2}$
    \item $\displaystyle\cos^2\frac{x}{2} = \frac{1+\cos x}{2}$
    \item $\displaystyle\tg^2\frac{x}{2} = \frac{1-\cos x}{1+\cos x}$
    \item $\displaystyle\ctg^2\frac{x}{2} = \frac{1+\cos x}{1-\cos x}$
    \item $\displaystyle\tg\frac{x}{2} = \frac{1-\cos x}{\sin x} = \frac{\sin x}{1+\cos x}$
    \item $\displaystyle\ctg\frac{x}{2} = \frac{1+\cos x}{\sin x} = \frac{\sin x}{1-\cos x}$
\end{itemize}

\subsection{Формулы квадратов тригонометрических функций}
\begin{itemize}
    \item $\displaystyle\sin^2x=\frac{1-\cos2x}{2}$
    \item $\displaystyle\cos^2x=\frac{1+\cos2x}{2}$
    \item $\displaystyle\tg^2x=\frac{1-\cos2x}{1+\cos2x}$
    \item $\displaystyle\ctg^2x=\frac{1+\cos2x}{1-\cos2x}$
    \item $\displaystyle\sin^2 \frac{x}{2}=\frac{1-\cos x}{2}$
    \item $\displaystyle\cos^2 \frac{x}{2}=\frac{1+\cos x}{2}$
    \item $\displaystyle\tg^2 \frac{x}{2}=\frac{1-\cos x}{1+\cos x}$
    \item $\displaystyle\ctg^2 \frac{x}{2}=\frac{1+\cos x}{1-\cos x}$
\end{itemize}

\subsection{Формулы кубов тригонометрических функций}
\begin{itemize}
    \item $\displaystyle\sin^3x = \frac{3\sin x-\sin3x}{4}$
    \item $\displaystyle\cos^3x = \frac{3\cos x+\cos3x}{4}$
    \item $\displaystyle\tg^3x = \frac{3\sin x-\sin3x}{3\cos x+\cos3x}$
    \item $\displaystyle\ctg^3x = \frac{3\cos x+\cos3x}{3\sin x-\sin3x}$
\end{itemize}

\subsection{Формулы тригонометрических функций в четвертой степени}
\begin{itemize}
    \item $\displaystyle\sin^4x=\frac{3-4\cos2x+\cos4x}{8}$
    \item $\displaystyle\cos^4x=\frac{3+4\cos2x+\cos4x}{8}$
\end{itemize}

\subsection{Формулы сложения аргументов}
\begin{itemize}
    \item $\displaystyle\sin(\alpha\pm\beta)=\sin\alpha\cos\beta\pm\cos\alpha\sin\beta$
    \item $\displaystyle\cos(\alpha\pm\beta)=\cos\alpha\cos\beta\mp\sin\alpha\sin\beta$
    \item $\displaystyle\tg(\alpha\pm\beta)=\frac{\tg\alpha\pm\tg\beta}{1\mp\tg\alpha\tg\beta}$
    \item $\displaystyle\ctg(\alpha\pm\beta)=\frac{\ctg\alpha\ctg\beta\mp1}{\ctg\alpha\pm\ctg\beta}$
\end{itemize}

\subsection{Формулы суммы тригонометрических функций}
\begin{itemize}
    \item $\displaystyle\sin\alpha+\sin\beta = 2\sin{\frac{\alpha+\beta}{2}}\cos{\frac{\alpha-\beta}{2}}$
    \item $\displaystyle\cos\alpha+\cos\beta = 2\cos{\frac{\alpha+\beta}{2}}\cos{\frac{\alpha-\beta}{2}}$
    \item $\displaystyle(\sin\alpha+\cos\alpha)^2=1+\sin2\alpha$
    \item $\displaystyle\tg\alpha+\tg\beta=\frac{\sin(\alpha+\beta)}{\cos\alpha\cos\beta}$
    \item $\displaystyle\ctg\alpha+\ctg\beta=\frac{\sin(\alpha+\beta)}{\sin\alpha\sin\beta}$
\end{itemize}

\subsection{Формулы разности тригонометрических функций}
\begin{itemize}
    \item $\displaystyle\sin\alpha-\sin\beta = 2\sin{\frac{\alpha-\beta}{2}}\cos{\frac{\alpha+\beta}{2}}$
    \item $\displaystyle\cos\alpha-\cos\beta = -2\sin{\frac{\alpha+\beta}{2}}\sin{\frac{\alpha-\beta}{2}}$
    \item $\displaystyle(\sin\alpha-\cos\alpha)^2=1-\sin2\alpha$
    \item $\displaystyle\tg\alpha-\tg\beta=\frac{\sin(\alpha-\beta)}{\cos\alpha\cos\beta}$
    \item $\displaystyle\ctg\alpha-\ctg\beta=-\frac{\sin(\alpha-\beta)}{\sin\alpha\sin\beta}$
\end{itemize}

\section{Производная}
\begin{itemize}
    \item $\displaystyle c' = 0,\ c=const$
    \item $\displaystyle x' = 1$
    \item $\displaystyle (x^n)'=nx^{n-1}$
    \item $\displaystyle \left(\frac{1}{x^n}\right)'=-\frac{n}{x^{n+1}}$
    \item $\displaystyle \left(\sqrt{x}\right)'=\frac{1}{2\sqrt x}$
    \item $(\sin x)'=\cos x$
    \item $(\cos x)'=-\sin x$
    \item $\displaystyle (\tg x)'=\frac{1}{\cos^2x}$
    \item $\displaystyle (\ctg x)'=-\frac{1}{\sin^2x}$
    \item $(e^x)'=e^x$
    \item $(a^x)'=a^x\ln a$
    \item $\displaystyle(\log_a x)' = \frac1{x\cdot \ln a}$
    \item $\displaystyle(\ln x)'=\frac 1x$
\end{itemize}

\subsection{Правила дифференцирования}
\begin{itemize}
    \item $\displaystyle(CU)' = CU'$
    \item $\displaystyle(U+V)'=U'+V'$
    \item $\displaystyle(UV)'=U'V+UV'$
    \item $\displaystyle\left(\frac UV\right)'=\frac{U'V-UV'}{V^2}$
    \item $\displaystyle\left(\frac 1V\right)'=-\frac{V'}{V^2}$
\end{itemize}

\section{Корень n-ной степени и его свойства}
\begin{itemize}
    \item $\displaystyle\sqrt[n]{ab} = \sqrt[n]{a}\sqrt[n]{b}$
    \item $\displaystyle\sqrt[n]{\frac ab} = \frac{\sqrt[n]{a}}{\sqrt[n]{b}},\ b\neq0$
    \item $\displaystyle\sqrt[n]{\sqrt[k]{a}} = \sqrt[nk]{a},\ k>0$
    \item $\displaystyle\sqrt[n]{a}=\sqrt[nk]{a^k},\ k>0$
    \item $\displaystyle\sqrt[n]{a^k}=\left(\sqrt[n]{a}\right)^k,\ k\leqslant0,a\neq0$
    \item $\displaystyle a^{\frac mn}=\sqrt[n]{a^m}$
\end{itemize}

\section{Первообразная}
\begin{quote}
Функция $F$ называется первообразной для функции $f$ на заданном промежутке, если для всех $x$ на этом промежутке $F'(x)=f(x)$
\end{quote}

\section{Интеграл}
\begin{itemize}
    \item{$\displaystyle\int\limits_a^b f(x)dx=F(b)-F(a)$}
    \item{$\displaystyle\left.\int\limits_a^b f(x)dx=F(x)\right|_a^b$}
\end{itemize}

\section{Логарифм}
\begin{itemize}
    \item $\displaystyle a^{\log_ab}=b,\ a\in(0;1)\cap(1;+\infty)$
    \item $\displaystyle\log_a1=0$
    \item $\displaystyle\log_aa=1$
    \item $\displaystyle\log_axy=\log_ax+\log_ay$
    \item $\displaystyle\log_a{\frac xy}=\log_ax-\log_ay$
    \item $\displaystyle\log_ax^p=p\log_ax$
    \item $\displaystyle\log_{a^n}{x^m} = \frac mn \log_ax$
    \item $\displaystyle\log_ab=\frac{\log_cb}{\log_ca}$
    \item $\displaystyle\log_ab=\frac{1}{\log_ba}$
\end{itemize}

\section{Прогрессии}

\subsection{Арифметическая прогрессия}
\begin{itemize}
    \item $\displaystyle x_{n+1}=x_n+d$
    \item $\displaystyle x_n=x_{n-1}+d=x_{n-2}+2d=\ldots=x_1+(n-1)d$
    \item $\displaystyle S_n=\frac{x_1+x_n}{2}n=\frac{2x_1+(n-1)d}{2}n$
    \item $\displaystyle x_n=\frac{x_{n-1}x_{n+1}}{2}$
    \item $\displaystyle x_1+x_n=x_2+x_{n-1}=x_3+x_{n-2}=\ldots$
\end{itemize}

\subsection{Геометрическая прогрессия}
\begin{itemize}
    \item $\displaystyle b_1=b_{n-1}q$
    \item $\displaystyle b_n=b_{n-1}q=b_{n-2}q^2=\ldots=b_1q^{n-1}$
    \item $\displaystyle S_n=\frac{b_1(1-q^n)}{1-q},\ q\neq1$
    \item $\displaystyle S=nq,\ q=1$
    \item $\displaystyle b^2_n=b_{n-1}b_{n+1}=\frac{b_n}{q}b_nq$
    \item $\displaystyle b_1b_n=b_2b_{n-1}=b_3b_{n-2}=\ldots$
\end{itemize}

\section{Комбинаторика}
\subsection{Размещение}
\begin{itemize}
  \item $\displaystyle A_n^k=\frac{n!}{(n-k)!}$
\end{itemize}
\subsection{Сочетание}
\begin{itemize}
  \item $\displaystyle C_n^k=\frac{n!}{k!(n-k)!}$
\end{itemize}


\end{document}
