\documentclass{proc}
\usepackage[utf8]{inputenc}
\usepackage[margin=0.5in]{geometry}
\usepackage{amsmath}
\usepackage[russian]{babel}

\title{Физика формулы}

\begin{document}
\pagestyle{empty}
\maketitle

\section{Механика}
\begin{enumerate}
    \item $\displaystyle V=\frac{S}{t}$ - скорость
    \item $\displaystyle x=x_0+V_{0x}t$ - уравнение движения
    \item $\displaystyle \vec{V_{a}}=\vec{V_{r}}+\vec{V_{e}}$ - закон сложения скоростей
    \item $\displaystyle V_{\textup{ср.}}=\frac{S_{\textup{вся}}}{t_{\textup{весь}}}$ - средняя скорость
    \item $\displaystyle \vec{a}=\frac{\vec{V}-\vec{V_0}}{t}$ - ускорение
    \item $\displaystyle \vec{V}=\vec{V_0}+\vec{a}t$ - скорость при равноускоренном движении
    \item $\displaystyle \vec{S}=\vec{V_0}t+\frac{\vec{a}t^2}{2}$ - путь
    \item $\displaystyle x=x_0+V_{0x}t+\frac{a_xt^2}{2}$
    \item $\displaystyle S_x=\frac{V_x^2-V_{0x}^2}{2a_x}$
    \item $\displaystyle V_y=V_{0y}+g_{y}t$
    \item $\displaystyle S_y=V_{0y}t+\frac{g_yt^2}{2}$
    \item $\displaystyle y=y_0+V_{0y}t+\frac{g_yt^2}{2}$
    \item $\displaystyle S_y=\frac{V_y^2-V_{0y}^2}{2}$
    \item $\displaystyle V_{ср}=\frac{V_0+V}{2}$ - при равноускоренном движении
    \item $\displaystyle a_{цс}=\frac{V^2}{R}=\omega^2R=\omega V$ - центростремительное ускорение
    \item $\displaystyle V=\frac{2\pi R}{T}=2\pi R\nu$ - линейная скорость при движении по окружности
    \item $\displaystyle \nu=n=\frac1T$ - частота
    \item $\displaystyle T=\frac tT$ - период
    \item $\displaystyle \nu = \frac NT$
    \item $\displaystyle \frac{V_1}{V_2}=\frac{R_1}{R_2}$, $\omega_1=\omega_2$ - отношение линейных скоростей двух тел к их расстоянию от оси вращения
    \item $\displaystyle T=\frac{t}{N}$; $\displaystyle \nu=\frac{N}{t}$; $\displaystyle T=\frac{1}{\nu}$ - колебания
    \item $\displaystyle \omega=2\pi \nu=\frac{2\pi}{T}$ - угловая скорость
    \item $\displaystyle V=\omega R$ -  линейная скорость
    \item $\displaystyle a_{x_{max}}=\omega^2x_{max}$ - максимальное линейное ускорение
    \item $\displaystyle V_{x_{max}}=\omega x_{max}$ - максимальная линейная скорость
    \item $\displaystyle F=mg$ - сила тяжести
    \item $\displaystyle F=kx$ - сила упругости
    \item $\displaystyle F=\mu N$ - сила реакции опоры
    \item $\displaystyle F = \rho gV_{пчт}$ - сила Архимеда для погруженной части тела
    \item $\displaystyle P=|\vec{N}|$; $P=|\vec{T}|$ - вес тела
    \item если $\vec{R}=0$, то $V=const$ - I закон Ньютона
    \item $\displaystyle \vec{F}=m\vec{a}$ - II закон Ньютона
    \item $\displaystyle \vec{F}_{12}=-\vec{F}_{21}$ - III закон Ньютона
    \item $\displaystyle F = G\frac{m_1m_2}{R^2}$ - закон всемирного тяготения
    \item $\displaystyle g = G\frac{M}{(R+h)^2}$ - ускорение свободного падения
    \item $\displaystyle V=\sqrt{gR}$ - 1 космическая скорость
    \item $\displaystyle \vec{p}=m\vec{V}$ - импульс тела
    \item $\displaystyle \vec{F}t=\Delta\vec{p}$ - изменение импульса
    \item $\displaystyle \Delta\vec{p}=m\vec{V}-m\vec{V_0}$
    \item $\displaystyle m_1\vec{V_1}+m_2\vec{V_2}=m_1\vec{V'_1}+m_2\vec{V'_2}$
    \item $\displaystyle A=FS\cos\alpha$ - работа силы
    \item $\displaystyle E_{\textup{к}}=\frac{mV^2}{2}$ - кинетическая энергия
    \item $\displaystyle E_{\textup{п}}=mgh$ - потенциальная энергия
    \item $\displaystyle E_{\textup{п}}=\frac{kx^2}{2}$ - потенциальная энергия упругого тела
    \item $\displaystyle E_{\textup{к}_1}+E_{\textup{п}_1}=E_{\textup{к}_2}+E_{\textup{п}_2}$ - закон сохранения энергии
    \item $\displaystyle A=E_{\textup{к}_2}-E_{\textup{к}_1}$ - теорема о кинетической энергии
    \item $\displaystyle A=-(E_{\textup{п}_2}-E_{\textup{п}_1})$ - теорема о потенциальной энергии
    \item $\displaystyle N=\frac At$ - мощность
    \item $\displaystyle N=FV\cos\alpha$
    \item $\displaystyle M=Fl$ - момент силы
    \item $\displaystyle \sum\limits_{i=1}^N M_i=0$ - условие равновесия через математическую сумму моментов
    \item $\displaystyle \sum\limits_{i=1}^N \vec{F}_i=0$ - условие равновесия через геометрическую сумму сил
    \item $\displaystyle p=\frac{F}{S}$ - давление
    \item $\displaystyle p=\rho gh$ - давление столба жидкости
    \item $\displaystyle \frac{F_1}{F_2}=\frac{l_2}{l_1}$ - рычаг
    \item $\displaystyle \frac{F_1}{F_2}=\frac{S_1}{S_2}$ - гидравлический пресс
    \item $\displaystyle x=x_m\cos\omega t$ - гармонические колебания
    \item $\displaystyle V=x'=-x_m\omega\sin\omega t$
    \item $\displaystyle a=V'=x''=-x_m\omega^2\cos\omega t$
    \item $\displaystyle m=\rho V$ - масса тела
    \item $\displaystyle \sum\limits_{i=1}^N \vec{F}_i = \vec{R}$ - равнодействующая сила
    \item $\displaystyle T=2\pi\sqrt{\frac{l}{g}}$ - математический маятник
    \item $\displaystyle T=2\pi\sqrt{\frac{m}{k}}$ - пружинный маятник
    \item $\displaystyle \omega_\textup{м.м.}=\sqrt{\frac gl}$
    \item $\displaystyle \omega_\textup{п.м.}=\sqrt{\frac km}$
\end{enumerate}

\section{Молекулярная физика, Термодинамика}
\begin{enumerate}
    \item $\displaystyle m=\rho V$ -  масса тела
    \item $\displaystyle m=m_0N$
    \item $\displaystyle m_0=\frac{M}{N_A}=\frac{\rho}{n}$ -  масса молекулы вещества
    \item $\displaystyle \nu=\frac mM=\frac{N}{N_A}$ -  количество вещества
    \item $\displaystyle p=\frac 13 m_0 n \bar{V}^2$ -  давление
    \item $\displaystyle p=\frac 13 \rho \bar{V}^2$
    \item $\displaystyle p=\frac 23 n \bar E$
    \item $\displaystyle p=nkT$
    \item $\displaystyle \bar E = \frac{m_0\bar V^2}{2}$ -  средняя кинетическая энергия молекулы
    \item $\displaystyle \bar E = \frac 32 kT$
    \item $\displaystyle N_A=6,022\cdot10^{23}$ $[$моль$^{-1}]$ - число Авогадро
    \item $\displaystyle k=1,38\cdot10^{-23}$ $[$Дж$\cdot$K$^{-1}]$ - постоянная Больцмана
    \item $\displaystyle R=N_Ak = 8,31$ - универсальная газовая постоянная
    \item $\displaystyle T=t_0+243\ [K]$ - связь между градусной шкалой Цельсия и абсолютной шкалой
    \item $\displaystyle pV=\frac{m}{M}RT$ - уравнение Менделеева-Клапейрона
    \item $\displaystyle \frac{p_1V_1}{T_1}=\frac{p_2V_2}{T_2}$ - уравнение Клапейрона
    \item $\displaystyle \frac{p_1}{p_2}=\frac{V_2}{V_1},\ T=cosnt$ - закон Бойля-Мариотта
    \item $\displaystyle \frac{p_1}{p_2}=\frac{T_1}{T_2},\ V=cosnt$ - закон Шарля
    \item $\displaystyle \frac{T_1}{T_2}=\frac{V_1}{V_2},\ p=cosnt$ - закон Гей-Люссака
    \item $\displaystyle \sum_{i=1}^N p_i = p_{смеси}$ - закон Дальтона
    \item $\displaystyle A = p\Delta V$ - работа
    \item $\displaystyle \Delta U = \frac 32 \nu R \Delta T$ - одноатомный газ
    \item $\displaystyle \Delta U = \frac 52 \nu R \Delta T$ -двухатомный газ
    \item $\displaystyle Q = \Delta U + A$ - первое начало термодинамики
    \item $\displaystyle A=-A'$ - равенство работ
    \item $\displaystyle \Delta U = Q + A'$ - изменение внутренней энергии газа
    \item $\displaystyle C = cm$ - теплоемкость
    \item $\displaystyle Q = cm\Delta t$ - количество теплоты при нагревании
    \item $\displaystyle Q = \lambda m$ - количество теплоты при плавлении
    \item $\displaystyle Q = qm$ - количество теплоты при сжигании топлива
    \item $\displaystyle Q = Lm$ - количество теплоты при испарении/кристаллизации
    \item $\displaystyle \sum_{i=1}^N Q_i = 0$ - уравнение теплового баланса
    \item $\displaystyle Q_\textup{отд}=Q_\textup{пол}$
    \item $\displaystyle \eta=\frac{Q_\textup{полезное}}{Q_\textup{полученное}}\cdot100\%=\frac{A}{Q_\textup{полученное}}\cdot100\%$ - КПД
    \item $\displaystyle \eta=\frac{T_\textup{н}-T_\textup{х}}{T_\textup{н}}\cdot100\%=\frac{Q_\textup{н}-Q_\textup{х}}{Q_\textup{н}}\cdot100\%$ - КПД тепловой машины
    \item $\displaystyle V=\sqrt{\frac{3kT}{m_0}}$ - средняя квадратичная скорость молекулы
    \item $\displaystyle \varphi=\frac{p}{p_\textup{н.п.}}\cdot100\%=\frac{\rho}{\rho_\textup{н.п.}}\cdot100\%$ - относительная влажность
\end{enumerate}

\section{Электричество}
\begin{enumerate}
    \item $\displaystyle q = Ne$ - заряд
    \item $\displaystyle F = k\frac{q_1q_2}{\varepsilon R^2}$ - сила Кулона
    \item $\displaystyle k = \frac{1}{4\pi\varepsilon_0}$
    \item $\displaystyle \vec E = \frac{\vec F}{q}$ - напряженность
    \item $\displaystyle E = k\frac{q}{\varepsilon r}$
    \item $\displaystyle W=k\frac{q_1q_2}{\varepsilon r}$ - электрическая энергия
    \item $\displaystyle \varphi=\frac{W}{q}$ - потенциал
    \item $\displaystyle \varphi=Er$
    \item $\displaystyle \varphi=k\frac{q}{\varepsilon r}$
    \item $\displaystyle U=\varphi_2-\varphi_1=\Delta U$ - разность потенциалов
    \item $\displaystyle U = \frac Aq$ - напряжение
    \item $\displaystyle A = q(\varphi_2-\varphi_1)=qU$
    \item $\displaystyle E = \frac Ud$
    \item $\displaystyle \sum_{i=1}^N \vec E_i = \vec E$
    \item $\displaystyle \sum_{i=1}^N \varphi_i = \varphi_{рез}$
    \item $\displaystyle \varepsilon=\frac{E_\textup{вакуум}}{E}$
    \item $\displaystyle C=\frac q\varphi$; $\displaystyle C = \frac qU$ - ёмкость конденсатора
    \item $\displaystyle C = \frac{\varepsilon_0\varepsilon S}{d}$
    \item $\displaystyle W_{эл} = \frac{qU}{2}=\frac{CU^2}{2}=\frac{q^2}{2C}$
    \item $\displaystyle I=\frac qt$ - сила тока
    \item $\displaystyle I = neVS$
    \item $\displaystyle j=\frac IS$ - плотность тока
    \item $\displaystyle I = \frac UR$ - Закон Ома
    \item $\displaystyle R = \rho \frac lS$
    \item последовательное соединение
        \begin{itemize}
            \item $\displaystyle I_1=I_2=\ldots=I_n$
            \item $\displaystyle U = U_1 + U_2+\ldots+U_n$
            \item $\displaystyle R = R_1 + R_2+\ldots+R_n$
            \item $\displaystyle L = L_1 + L_2+\ldots+L_n$
            \item $\displaystyle \frac1C=\frac1{C_1}+\frac1{C_2}+\ldots+\frac1{C_n}$
        \end{itemize}
    \item параллельное соединение
        \begin{itemize}
            \item $\displaystyle U_1=U_2=\ldots=U_n$
            \item $\displaystyle I = I_1 + I_2 + \ldots + I_n$
            \item $\displaystyle \frac{1}{R}=\frac{1}{R_1}+\frac{1}{R_2}+\ldots+\frac1{R_n}$
            \item $\displaystyle \frac1L=\frac1{L_1}+\frac1{L_2}+\ldots+\frac1{L_n}$
            \item $\displaystyle C = C_1 + C_2+\ldots+C_n$
        \end{itemize}
    \item $\displaystyle I = \frac{\varepsilon}{R+r}$
    \item $\displaystyle \varepsilon=U_R+U_r$ - ЭДС
    \item $\displaystyle I=\frac{\varepsilon}{r}$ - ток короткого замыкания
    \item $\displaystyle \varepsilon = \frac{Ac}{q}$
    \item $\displaystyle A = UIt = \frac{U^2t}{R}=I^2Rt$ - работа силы тока
    \item $\displaystyle Q=I^2Rt$ - закон Джоуля-Ленца
    \item $\displaystyle \eta=\frac U\varepsilon\cdot100\%$ - КПД источника
    \item $\displaystyle P = UI = I^2R = \frac{U^2}{R}$ - мощность
    \item $\displaystyle m=kIt$
    \item $\displaystyle \varepsilon=IR+I_2$
\end{enumerate}

\section{Магнетизм}
\begin{enumerate}
    \item $\displaystyle B = \frac{F_{max}}{Il}$ - магнитная индукция
    \item $\displaystyle F_A=BIl\sin\alpha$
    \item $\displaystyle F_A=qVB\sin\alpha$
    \item $\displaystyle \Phi=BS\cos\alpha$
    \item $\displaystyle \varepsilon_i=-\frac{\Delta\Phi}{\Delta t}N$
    \item $\displaystyle \varepsilon_i=BVl\sin\alpha$
    \item $\displaystyle \varepsilon_{iS}=-L\frac{\Delta I}{\Delta t}$
    \item $\displaystyle \Phi = LI$
    \item $\displaystyle W_{маг}=\frac{LI^2}{2}$
\end{enumerate}


\section{Электромагнитные колебания и волны}
\begin{enumerate}
    \item $\displaystyle \lambda = \frac V\nu=VT$
    \item $\displaystyle T = 2\pi\sqrt{LC}$
    \item $\displaystyle X_C=\frac{1}{\omega C}$
    \item $\displaystyle X_L = \omega L$
    \item $\displaystyle U=\frac{U_m}{\sqrt{2}}$; $\displaystyle I=\frac{U_m}{\sqrt{2}}$
    \item $\displaystyle \omega=\frac{1}{\sqrt{LC}}$
    \item $\displaystyle k=\frac{U_1}{U_2}=\frac{N_1}{N_2}=\frac{I_2}{I_1}$
\end{enumerate}

\section{Оптика}
\begin{enumerate}
    \item $\displaystyle n = \frac{V_1}{V_2}=\frac{n_2}{n_1}$ - коэффициент преломления
    \item $\displaystyle \angle\alpha = \angle\beta$ - угол падения равен углу отражения
    \item $\displaystyle \frac{\sin\alpha}{\sin\beta}=\frac{n_2}{n_1}=n$
    \item $\displaystyle n_1 = \frac{c}{V_1}$; $\displaystyle n_2 = \frac{c}{V_2}$
    \item $\displaystyle \pm\frac 1F=\pm\frac 1d \pm\frac 1f$ - формула тонкой линзы
    \item $\displaystyle D = \frac 1F$ - диоптрийная сила
    \item $\displaystyle \Gamma=\frac Hh = \frac fd$ - линейное увеличение
    \item $\displaystyle D = D_1+D_2$
    \item $\displaystyle \Gamma = \Gamma_1\cdot\Gamma_2$
    \item $\displaystyle n = \frac{V_1}{V_2}=\frac{\lambda_1}{\lambda_2}$
    \item $\displaystyle k\lambda = \Delta d$ - условие максимума
    \item $\displaystyle (2k+1)\lambda = \Delta d$ - условие минимума
    \item $\displaystyle k\lambda = d\sin\varphi$
    \item $\displaystyle d = \frac lN$
\end{enumerate}

\section{Квантовая и ядерная физика}
\begin{enumerate}
    \item $\displaystyle E = h\nu = \frac{hc}{\lambda}$
    \item $\displaystyle h\nu = A_{\textup{вых.}}+E_k$ - уравнение Эйнштейна для фотоэффекта
    \item $\displaystyle A=h\nu_{\textup{кр.}}=\frac{hc}{\lambda_{\textup{кр.}}}$
    \item $\displaystyle E\left(_Z^AX\right)=931(Z\cdot m_p+(A-Z)\cdot m_n+M)$ - энергия связи
    \item $\displaystyle \varepsilon=\frac{E}{A}$
    \item $\displaystyle N=N_0\cdot2^{-\frac tT}$ - закон распада
    \item $\displaystyle m=m_0\cdot2^{-\frac tT}$
    \item $\displaystyle _Z^AX\ \rightarrow\ _{Z-2}^{A-4}Y\ +\ _2^4He$ - $\alpha$-распад
    \item $\displaystyle _Z^AX\ \rightarrow\ _{Z+1}^{A}Y+\ _{-1}^0e$ - $\beta$-распад
    \item $\displaystyle E=mc^2$
\end{enumerate}

\section{СТО}
\begin{enumerate}
    \item $\displaystyle E=mc^2$
    \item $\displaystyle V = \frac{V_0+V_1}{1+\frac{V_0V_1}{c^2}}$
    \item $\displaystyle t = \frac{t_0}{\sqrt{1-\frac{V^2}{c^2}}}$
    \item $\displaystyle l = l_0\sqrt{1-\frac{V^2}{c^2}}$
    \item $\displaystyle p=\frac{EV}{c^2}$
\end{enumerate}

\end{document} 